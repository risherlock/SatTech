\documentclass{article}
\usepackage[margin=1in]{geometry}
\usepackage{amsmath, amssymb, bm, amsthm}
\usepackage{enumitem, hyperref}
\usepackage[breakable, skins]{tcolorbox}

\newtcolorbox{ans_box}[0]{arc=0mm, enhanced, frame hidden, breakable}

\setlength{\parindent}{0pt}
\setlength{\parskip}{0.5em}

\title{Exercises in Lie Theory}
\date{November 5, 2024}
\author{Rishav}

\begin{document}

\maketitle
\hrule
\tableofcontents
\newpage

\section{Topology}

\begin{enumerate}
  \item Give an example that in a general a family of open sets is not closed under infinite intersections.

  \begin{ans_box}
    Let us consider $\mathbb{R}$ under the metric d$(x,y)=|x-y|$ with open sets $U_{n}=\left(-\dfrac{1}{n},\dfrac{1}{n}\right)$.
    $$
    \bigcap_{n=1}^{\infty}U_{n}=\bigcap_{n=1}^{\infty}\left(-\dfrac{1}{n},\dfrac{1}{n}\right)=\{0\}
    $$

  Intersection of the open sets $U_{n}$ becomes narrower with increase in $n$ and approaches $\{0\}$ as $n\rightarrow\infty$.\medskip

  We know that a subset $U\subset\mathbb{R}$ is said to be open if, for every point $x\in U$, there exists an $\varepsilon>0$ such that the open ball $B(x,\varepsilon)=\{y\in\mathbb{R}\,|\,\text{d}(x,y)<\varepsilon\}$. We cannot create such open ball within a singleton $\{0\}$. Therefore $U_{n}$ is not closed under infinite intersection.
  \end{ans_box}

  \item Give an example that a bijective continuous function $f:E\rightarrow F$ between topological spaces is not necessarily a homeomorphism.

  \begin{ans_box}
    Let us consider that $E=\mathbb{R}$ has the standard topology generated by open intervals and $F=\mathbb{R}$ has the lower limit topology generated by basis of half-open intervals $[a,b)$ for $a<b$ where $a,b\in\mathbb{R}$. Furthermore, we define the function $f:E\rightarrow F$ as identity map i.e. $f(x)=x$.\medskip

  $f$, being an identity map, is clearly a bijection and also continuous because open sets in $E$ is also an open set in $F$. However, $f^{-1}:F\rightarrow E$ is not continuous because not every pre-image of open sets in $F$ is an open set in $E$ under $f$. For example, $[0,1)\in F$ is not an open set in $E$.
  \end{ans_box}

  \item Give an example of connected topological space which is not path connected.

  \begin{ans_box}
    Consider
    \begin{equation*}
      \begin{split}
        Y&=\{(0,y)\in\mathbb{R}^{2}|y\in[-1,1]\},\\
        Z&=\{(x,\sin\frac{\pi}{x})\in\mathbb{R}^{2}|x\in(0,1]\}, \text{ and}\\
        X&=Y\cup Z.
      \end{split}
    \end{equation*}

    Euclidean topology is induced on $X$ by inclusion on $\mathbb{R}^{2}$. \medskip

    $Z$ is connected space because it is an image of connected set $(0,1]$ under continuous sine function. $X$ is closure of $Z$ in $\mathbb{R}^{2}$ which makes $X$ a connected space. \medskip

    Let $\gamma:[0,1]\mapsto X$ be a path in $X$ that begins at $y\in Y$ and ends at $z\in Z$ with $\gamma(0)=y$ and $\gamma(1)=z$. Existence of such a map $\gamma$ would make $X$ a path-connected space. This can be contradicted if we show that $\gamma([0,1])\subseteq Y$ i.e. path is confined to $Y$.\medskip

    $\gamma^{-1}(Y)$ is certainly nonempty as it contains at least one element i.e. $0$. Suppose $t\in\gamma^{-1}(Y)$ and $\varepsilon>0$ with a value which ensure $\gamma\left((t-\varepsilon,\,t+\varepsilon)\right)$ is contained in a closed disc $D$ centered at $\gamma(t)$ and radius $r$. The intersection of this disc with $X$ consists of two intervals; i) closed interval in $Y$ and ii) closed segment of curve in $Z$ i.e. $$D\cup X=(D\cap Y)\cup(D\cap Z).$$

    We know that $(D\cap Y)\cap(D\cap Z)=\emptyset$, therefore $(D\cap Y)$ and $(D\cap Z)$ are separated from one another. Since $\gamma(t)\in D\cap Y$ and $(t-\varepsilon,\,t+\varepsilon)$ is connected, the image should be connected as well. This forces all the elements of $\gamma\left((t-\varepsilon,\,t+\varepsilon)\right)$ to be in $(D\cap Y)$ which implies $\gamma([0,1])\subseteq Y$.
  \end{ans_box}

\item  Give an example of a connected topological space which is not locally connected.

\begin{ans_box}
  Let us take connected space $X$ in above exercise. To show that $X$ is not locally connected, we have to find at least a point in $p\in X$ that does not contain a connected open neighborhood.

  Suppose an open disc centered on $p=(0,1)\in Y\in X$ with radius $r$. Since $Y$ is a closed set and $p$ is the edge of the line, there is no open connected open neighborhood of $p$.
\end{ans_box}

\item Recall the

  \textbf{Definition.} A topological space $X$ is \textit{totally disconnected} if the connected components of $X$ are single points.

  Give an example of a totally disconnected topological space.

  \begin{ans_box}
    Suppose $A=\{x\in\mathbb{Q}|x<\sqrt{2}\}=(-\infty,\sqrt{2})\cap\mathbb{Q}$ where the open segment $(-\infty,\sqrt{2})\in\mathbb{R}$ is open in $\mathbb{R}$ so $A$ is open in the subspace topology on $\mathbb{Q}$. Similarly $B=\{x\in\mathbb{Q}|x>\sqrt{2}\}=(\sqrt{2},+\infty,)\cap\mathbb{Q}$ is open in $\mathbb{Q}$ too. These open sets are disjoint and their union is $\mathbb{Q}$, therefore $\mathbb{Q}$ is not connected.\medskip

  Between any two rational numbers there exists an irrational number. Therefore $x,y\in\mathbb{Q}$ belong to different components of $\mathbb{Q}$. Indeed the components of $\mathbb{Q}$ are  singletons which, by definition, implies that $\mathbb{Q}$ is totally disconnected.
  \end{ans_box}
  \item Recall the

  \textbf{Definition.} Let $X$ be a topological space. A subset $S\subset X$ is said to be \textit{dense} in $X$ if, for each $x\in X$ and each neighborhood $U$ of $x$, there is a point $s\in S$ s.t. $s\in U$.

  Give an example of a dense subset of $\mathbb{R}$ (with usual topology.)

  \item Let $X$ be the space of continuous functions on the interval $I:=[0,1]$. Show that $$\text{d}(f,g):=\max_{x\in I}|f(x)-g(x)|$$ defines a metric on $X$.
\end{enumerate}

\section{Differential calculus}

\begin{enumerate}[start=9]
  \item Add to your personal mathematical tool box the following
  \paragraph{Definition.}

  \begin{enumerate}
    \item Let $E$ be a normed space and $f:U\subset E\rightarrow\mathbb{R}$ be differentiable so that D$f(u)\in L(E,\mathbb{R})=E^{*}$. In this case we sometimes write d$f(u)$ for D$f(u)$ and call d$f$ the \textit{differential} of $f$. Thus d$f:U\rightarrow E^{*}$.

    \item Let $(E, \langle.,.\rangle)$ be a Hilbert space and $f:U\subset E\rightarrow\mathbb{R}$ be differentiable. The \textit{gradient} of $f$ is the map grad$f=\nabla f:U\rightarrow E$ defined (implicitly) by $\langle\nabla f(u), e\rangle:=\text{d}f(u)\cdot e$, meaning the linear map d$f(u)$ applied to the vector $e$ (directional derivative).
  \end{enumerate}

  \item

  \item Endow the vector space $\mathbb{R}^{n\times n}$ (matrix space) with the inner product $\langle A,B\rangle:=\text{tr}(A^{\top}B)$. Consider the smooth function $f:\text{GL}(\mathbb{R}^{n})\rightarrow\mathbb{R}$ defined by $X\mapsto\text{tr}(X^{-1})$ and compute grad$f(X)$. (

  {\footnotesize Hint: Do not even try to work with partial derivatives.}

  \item For $g:\mathbb{R}^{n\times n}\rightarrow\mathbb{R}^{n\times n}$, defined by $X\mapsto X^{3}$, for $H\in\mathbb{R}^{n\times n}$, compute the directional derivative D$g(X)\cdot H$.

  {\footnotesize Hint: Ditto.}
\end{enumerate}

\section{Smooth manifolds}

\begin{enumerate}[start=13]
  \item
  \item
  \item
  \item \label{item:16} Show that the set GL$_{n}(\mathbb{R})$ of real invertible $n\times n$ matrices is a differentiable manifold.

  \item Similar to Exercise \ref{item:16} show that the set $\mathbb{R}^{n\times m}_{*}$, i.e., the set of all full rank real rectangular matrices with $n\leq m$ forms a $nm$-dimensional submanifold of the vector space $\mathbb{R}^{n\times m}\cong\mathbb{R}^{nm}$.
\end{enumerate}

\section{Tangent structures}

\begin{enumerate}[start=19]
  \item
  \item Let
  $$
  g:\mathbb{R}^{2}\rightarrow\mathbb{R}^{3},\quad
  \begin{bmatrix}x\\y\end{bmatrix}\mapsto
  \begin{bmatrix}x^{2}y+y^{2}\\x-2y^{3}\\ye^{x}\end{bmatrix}.
  $$

  \begin{enumerate}
    \item Compute $T_{\left[\begin{smallmatrix}x\\y\end{smallmatrix}\right]}g$ and $T_{\left[\begin{smallmatrix}0\\0\end{smallmatrix}\right]}g\cdot\left(4\dfrac{\partial}{\partial x}-\frac{\partial}{\partial y}\right)$ via the algebraic approach.
    \item Calculate the conditions that the constants $\lambda,\mu,\nu\in\mathbb{R}$ must satisfy for the vector
    $$
    \left(\lambda\frac{\partial}{\partial x}+\mu\frac{\partial}{\partial y}+\nu\frac{\partial}{\partial z}\right)
    \Bigg|_{g\left(\left[\begin{smallmatrix}0\\0\end{smallmatrix}\right]\right)}
    $$
    to be the image of some vector by $Tg$.
  \end{enumerate}

  \item Consider
  $$
  f:\mathbb{R}^{2}\rightarrow\mathbb{R},\quad
  \begin{bmatrix}x\\y\end{bmatrix}\mapsto
  x^{3}+xy+y^{3}+1.
  $$

  \begin{enumerate}
    \item Using the algebraic approach compute the tangent map
    $$
    Tf:T_{\left[\begin{smallmatrix}x\\y\end{smallmatrix}\right]}\mathbb{R}^{2}
    \rightarrow T_{f\left(\left[\begin{smallmatrix}x\\y\end{smallmatrix}\right]\right)}\mathbb{R}.
    $$

    \item Which of the points
    $p\in\left\{\begin{bmatrix}0\\0\end{bmatrix}, \begin{bmatrix}1/3\\1/3\end{bmatrix}, \begin{bmatrix}-1/3\\-1/3\end{bmatrix}\right\}$
    is $Tf$ injective or surjective at?
  \end{enumerate}

  \item Consider the smooth map
  $$
  f_{\theta}:\mathbb{R}^{2\times2}\rightarrow\mathbb{R}^{2\times2},\quad
  \begin{bmatrix}x&z\\y&t\end{bmatrix}=:X\mapsto A_{\theta}\cdot X,\quad\text{with}\quad A_{\theta}:=
  \begin{bmatrix}\cos\theta&-\sin\theta\\\sin\theta&\cos\theta\end{bmatrix}.
  $$

  \begin{enumerate}
    \item Compute the tangent map T$f_{\theta}$ via the algebraic approach.

    {\footnotesize Hint: It may be easier to proceed by first identifying $\mathbb{R}^{2\times2}\cong\mathbb{R}^{4}$ in a suitable way, but one certainly can go ahead without doing so, as well.}

    \item Compute $Tf_{\theta}V$ with
    $$
    V:=\cos\theta\frac{\partial}{\partial x}-\sin\theta\frac{\partial}{\partial y}+
    \cos\theta\frac{\partial}{\partial z}-\cos\theta\frac{\partial}{\partial t}.
    $$
  \end{enumerate}

  \item

  \item Consider the differentiable map
  $$
  \varphi:\mathbb{R}^{4}\rightarrow\mathbb{R}^{2},\quad
  \begin{bmatrix}x\\y\\z\\t\end{bmatrix}\mapsto
  \begin{bmatrix}x^{2}+y^{2}+z^{2}+t^{2}-1\\x^{2}+y^{2}+z^{2}+t^{2}-2y-2z+5\end{bmatrix}=:
  \begin{bmatrix}u\\v\end{bmatrix}
  $$

  \begin{enumerate}
    \item Find the (i) set of points of $\mathbb{R}^{4}$ where $\varphi$ is not a submersion, and (ii) its image.
    \item Compute a basis for ker$T\varphi\Big|_{[0\,1\,2\,0]^{\top}}$.
    \item Compute the image of the tangent vector $[0\,1\,2\,0]^{\top}\in T_{[0\,1\,2\,0]^{\top}}\mathbb{R^{4}}$ by the tangent of $\varphi$.
    \item Compute the image of the covector $(\text{d}u+2\text{d}v)\Big|_{\begin{smallmatrix}-1\\5\end{smallmatrix}}\in T^{*}_{\begin{smallmatrix}-1\\5\end{smallmatrix}}\mathbb{R}^{2}$ by the pull-back $\phi^{*}$, choosing point $\begin{bmatrix}0&0&0&0&0\end{bmatrix}^{\top}\in\varphi^{-1}\left\{\begin{bmatrix}-1\\5\end{bmatrix}\right\}.$
  \end{enumerate}

  \item Consider the differentiable map
  $$
  X:=xy\frac{\partial}{\partial x}+x^{2}\frac{\partial}{\partial z},\quad Y:=y\frac{\partial}{\partial y},
  $$
  and the map $f:\mathbb{R}^{3}\rightarrow\mathbb{R}$, defined by $f(x,y,z):=x^{2}y$. Compute the followings for $p=\begin{bmatrix}1\\1\\0\end{bmatrix}$

  \begin{enumerate}
    \item $[X,Y]\Big|_{p}$
    \item $[fX]\Big|_{p}$
    \item $(Xf)\Big|_{p}$
    \item $Tf\cdot X|_{p}$
  \end{enumerate}

\end{enumerate}

\section{Lie theory}

\end{document}