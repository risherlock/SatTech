\documentclass{article}
\usepackage[margin=0.8in]{geometry}
\usepackage{graphicx, hyperref, amsmath, amssymb, amsthm}

\setlength{\parindent}{0pt}
\setlength{\parskip}{0.5em}


\title{Introduction to Differential Geometry}
\author{based on lectures\\by Prof. Stefan Waldmann\vspace{1.5in}\\Rishav}
\date{October 14, 2024}

\begin{document}
\maketitle

\theoremstyle{definition}
\newtheorem{definition}{Definition}[section]
\newtheorem{example}{Example}[section]
\newtheorem{theorem}{Theorem}[section]
\newtheorem{lemma}{Lemma}[section]
\newtheorem{remark}{Remark}[section]
\newtheorem{proposition}{Proposition}[section]
\newtheorem{corollary}{Corollary}[section]

\newpage
\tableofcontents
\newpage

\section{Differentiable manifolds}

\subsection{Introduction}

\subsection{Topology}

\begin{definition}[Topological space]
\end{definition}

\begin{definition}[Second countable]
\end{definition}

\begin{example}[$\mathbb{R}^{n}$ is second countable]
\end{example}

\begin{definition}[Continuity]
\end{definition}

\begin{example}[Topological spaces]
\end{example}

\begin{definition}[Homeomorphism of embeddings]
\end{definition}

\begin{example}[Continuous and bijective map with discontinuous inverse]
\end{example}

\subsection{Connectedness and separation properties}

\begin{definition}[Conectedness]
\end{definition}

\begin{example}[Connected subsets of $\mathbb{R}$]
\end{example}

\begin{proposition}[Continuous images of connected spaces are connected.]
\end{proposition}

\begin{definition}[Path connectedness]
\end{definition}

\begin{proposition}[Path connected implies connected]
\end{proposition}

\begin{example}[Connected but not pathconnected]
\end{example}

\begin{definition}[Locally path connected]
\end{definition}

\begin{example}[$\mathbb{R}^{n}$ is locally path connected]
\end{example}

\begin{definition}[Seperation axioms]
\end{definition}

\subsubsection{Initial and final topologies}

\begin{definition}[Product topology]
\end{definition}

\begin{definition}[Quotient topology]
\end{definition}

\subsubsection{Compactness}

\begin{definition}[Compactness]
\end{definition}

\begin{proposition}[Properties of compact sets]
\end{proposition}

\begin{proposition}[Image of a compact set under continuous map is compact]
\end{proposition}

\begin{corollary}[]
\end{corollary}

\begin{theorem}[Tikhonov's theorem]
\end{theorem}

\begin{remark}[Compaceness implies sequential compactness for second countable spaces]
\end{remark}

\begin{definition}[Locally compact spaces]
\end{definition}

\begin{proposition}[Locally compact Hausdroff space implies $T_{3}$]
\end{proposition}

\begin{proposition}
\end{proposition}

\subsection{From topological space to differentiable manifolds}

\subsubsection{Topological manifolds}

\begin{definition}[Topological atlas]
\end{definition}

\begin{remark}[Does the existance of atlas imply Hausdroff?]
\end{remark}

\begin{definition}[Topological manifold]
\end{definition}

\begin{proposition}[Free features of topological manifolds]
\end{proposition}

\begin{proposition}[Inheritance from $\mathbb{R}^{n}$]
\end{proposition}

\begin{corollary}
\end{corollary}

\begin{remark}[Product of manifolds]
\end{remark}

\begin{remark}[Non-empty open subset of a topological manifold is also a topological manifold]
\end{remark}

\begin{definition}[Differentiable atlas and differentiable structure]
\end{definition}

\begin{proposition}
\end{proposition}

\begin{remark}
\end{remark}

\begin{remark}
\end{remark}

\subsubsection{Differentiable manifolds}

\begin{definition}[Differentiable manifolds]
\end{definition}

\begin{definition}[Complex manifold]
\end{definition}

\begin{proposition}
\end{proposition}

\begin{remark}
\end{remark}

\begin{remark}
  There are $\mathcal{C}^{0}$ topological manifolds which do not admit a $\mathcal{C}^{1}$ structure.
\end{remark}

\begin{example}[$\mathcal{S}^{7}$ has 28 equivalent smooth structures.]
\end{example}

\begin{example}[]
\end{example}

\begin{proposition}[Finite cartesian product of $\mathcal{C}^{k}$ manifolds]
\end{proposition}

\begin{example}[Real projective space]
\end{example}

\subsubsection{Differentiable maps}

\begin{lemma}[Continuous map at a point]
\end{lemma}

\begin{definition}[Continuous map]
\end{definition}

\begin{proposition}[Composition]
\end{proposition}

\begin{definition}[Holomorphic map]
\end{definition}

\begin{proposition}[Composition of $\mathcal{C}^{k}$ maps is $\mathcal{C}^{k}$]
\end{proposition}

\begin{proposition}
\end{proposition}

\begin{definition}[Diffeomorphism]
\end{definition}

\begin{remark}[Diffeomorphism of a manifold forms a group]
\end{remark}

\begin{remark}[Category manifold]
\end{remark}

\subsubsection{Differentiable functions and partitions of unity}

\begin{definition}[Differentiable functions]
\end{definition}

\begin{proposition}[]
\end{proposition}

\begin{proposition}
\end{proposition}

\begin{definition}[Compactly supported functions]
\end{definition}

\begin{proposition}
\end{proposition}

\begin{theorem}[Partition of unity]
\end{theorem}

\begin{definition}[Partition of unity]
\end{definition}


\begin{corollary}($\mathcal{C}^{\infty}$ Urysohn lemma)
\end{corollary}

\begin{corollary}
\end{corollary}


\subsection{Tangent bundle and the tangent map}

\subsubsection{Tangent vectors and tangent spaces}

\begin{definition}[Germ]
\end{definition}

\begin{proposition}
\end{proposition}

\begin{definition}[Tanget space]
\end{definition}

\begin{theorem}[Real tangent space]
\end{theorem}

\begin{remark}[Coordinate based definition of tangent vectors]
\end{remark}

\begin{remark}[Open sets in $\mathbb{R}^{n}$ as manifold]
\end{remark}

\begin{theorem}[Holomorphic tangent spaces]
\end{theorem}

\begin{remark}[Relation between real and holomorphic tangent vector for complex manifold]
\end{remark}

\begin{remark}[Einstein convention]
\end{remark}

\subsubsection{Tangent bundle}

\begin{definition}[Bundle chart for tangent bundle]
\end{definition}

\begin{lemma}[Cooking up a topology]
\end{lemma}

\begin{theorem}
\end{theorem}

\begin{theorem}[Tangent bundle]
\end{theorem}

\begin{theorem}[Holomorphic tangent bundle]
\end{theorem}

\begin{definition}[Tangent bundle]
\end{definition}

\begin{remark}[]
\end{remark}

\subsubsection{The tangent maps}

\begin{definition}[Pull-back]
\end{definition}

\begin{proposition}[1.3.17]
\end{proposition}

\begin{remark}[Milnor's exercise]
\end{remark}

\begin{theorem}[Tangent map]
\end{theorem}

\begin{remark}[Tangent functor]
\end{remark}

\begin{remark}[]
\end{remark}

\subsection{Sub manifolds}

\subsubsection{Immersion and submersion}

\begin{definition}[Rank]
\end{definition}

\begin{proposition}[1.4.2]
\end{proposition}

\begin{corollary}[1.4.3]
\end{corollary}

\begin{definition}[Regular points and regular values]
\end{definition}

\begin{definition}[Immersion and submersion]
\end{definition}

\begin{example}[1.4.8]
\end{example}

\begin{theorem}[Constant rank maps are boring]
\end{theorem}

\begin{corollary}[1.4.8]
\end{corollary}

\begin{corollary}[1.4.9]
\end{corollary}

\begin{remark}[Surjective submersion]
\end{remark}

\begin{example}[Determinant 1.4.11]
\end{example}

\begin{example}[Constant rank map]
\end{example}

\begin{proposition}[1.4.13]
\end{proposition}

\subsubsection{Submanifolds and embeddings}

\begin{definition}[Submanifold]
\end{definition}

\begin{proposition}[1.4.15]
\end{proposition}

\begin{proposition}[1.4.17]
\end{proposition}

\begin{proposition}[1.4.18]
\end{proposition}

\begin{definition}[Embedding]
\end{definition}

\begin{proposition}[1.4.21]
\end{proposition}

\begin{example}[1.4.22]
\end{example}

\begin{definition}[Transverse map]
\end{definition}

\begin{proposition}[1.4.25]
\end{proposition}

\begin{proposition}[1.4.26]
\end{proposition}

\begin{definition}[Fiber product]
\end{definition}

\begin{proposition}[1.4.28]
\end{proposition}

\begin{theorem}[Whitney]
\end{theorem}

\newpage
\section{Vector bundles and other sections}

\subsection{Vector bundles and first construction}

\subsubsection{Vector bundle charts and vector bundles}

\begin{definition}[Vector bundle action]
\end{definition}

\begin{definition}[Vector bundle]
\end{definition}

\begin{proposition}[2.1.18]
\end{proposition}

\begin{example}[2.1.4]
\end{example}

\begin{example}[2.1.5 Trivial vector bundle]
\end{example}

\begin{example}[Möbius strip]
\end{example}

\begin{example}[2.1.1]
\end{example}

\begin{lemma}[2.1.9]
\end{lemma}

\begin{proposition}[The category vectory]
\end{proposition}

\begin{proposition}
\end{proposition}

\begin{remark}[2.1.12]
\end{remark}

\begin{definition}[Equivalend cocycles]
\end{definition}

\begin{corollary}[2.1.15]
\end{corollary}

\begin{remark}[2.1.17]
\end{remark}

\begin{remark}[2.1.18]
\end{remark}

\subsection{Construction of vector bundles}

\begin{definition}[Vector subbundles]
\end{definition}

\begin{proposition}[2.1.20]
\end{proposition}

\begin{proposition}[2.1.21]
\end{proposition}

\begin{proposition}[2.1.22]
\end{proposition}

\begin{remark}[2.1.24]
\end{remark}

\begin{proposition}[2.1.26]
\end{proposition}

\begin{remark}[2.1.27]
\end{remark}

\begin{remark}[2.1.29]
\end{remark}

\end{document}
